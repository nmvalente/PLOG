\documentclass[a4paper]{article}

%use the english line for english reports
%usepackage[english]{babel}
\usepackage[portuguese]{babel}
\usepackage[utf8]{inputenc}
\usepackage{indentfirst}
\usepackage{graphicx}
\usepackage{verbatim}

\begin{document}

\setlength{\textwidth}{16cm}
\setlength{\textheight}{22cm}

\title{\Huge\textbf{Dominup em Prolog}\linebreak\linebreak\linebreak
\Large\textbf{Relatório Intercalar}\linebreak\linebreak
\linebreak\linebreak
\includegraphics[scale=0.1]{feup-logo.png}\linebreak\linebreak
\linebreak\linebreak
\Large{Mestrado Integrado em Engenharia Informática e Computação} \linebreak\linebreak
\Large{Programação em Lógica}\linebreak
}

\author{\textbf{Dominup 04:}\\
Ângela Filipa Pereira Cardoso - up200204375 \\
Nuno Miguel Rainho Valente - up200204376 \\
\linebreak\linebreak \\
 \\ Faculdade de Engenharia da Universidade do Porto \\ Rua Roberto Frias, s\/n, 4200-465 Porto, Portugal \linebreak\linebreak\linebreak
\linebreak\linebreak\vspace{1cm}}

\maketitle
\thispagestyle{empty}

%************************************************************************************************
%************************************************************************************************

\newpage

%Todas as figuras devem ser referidas no texto. %\ref{fig:codigoFigura}
%
%%Exemplo de código para inserção de figuras
%%\begin{figure}[h!]
%%\begin{center}
%%escolher entre uma das seguintes três linhas:
%%\includegraphics[height=20cm,width=15cm]{path relativo da imagem}
%%\includegraphics[scale=0.5]{path relativo da imagem}
%%\includegraphics{path relativo da imagem}
%%\caption{legenda da figura}
%%\label{fig:codigoFigura}
%%\end{center}
%%\end{figure}
%
%
%\textit{Para escrever em itálico}
%\textbf{Para escrever em negrito}
%Para escrever em letra normal
%``Para escrever texto entre aspas''
%
%Para fazer parágrafo, deixar uma linha em branco.
%
%Como fazer bullet points:
%\begin{itemize}
	%\item Item1
	%\item Item2
%\end{itemize}
%
%Como enumerar itens:
%\begin{enumerate}
	%\item Item 1
	%\item Item 2
%\end{enumerate}
%
%\begin{quote}``Isto é uma citação''\end{quote}


%%%%%%%%%%%%%%%%%%%%%%%%%%
\section{O Jogo Dominup}

Dominup é uma variação do jogo Dominó para 2 a 4 jogadores, em que, tal como o nome sugere, é possível colocar peças em cima de outras.

No típico Dominó existem 28 peças duplas numeradas de 0 a 6, à semelhança das faces de um dado. Já no Dominup há 36 peças duplas numeradas de 0 a 7, usando códigos binários: o ponto no centro representa 1, o circulo pequeno representa 2 e o circulo grande representa 4, como se pode ver na figura~\ref{piece}. Este desenho das peças, juntamente com as regras do Dominup e de dois outros jogos, foram criadas por Néstor Romeral Andrés em 2014, sendo o conjunto publicado por nestorgames\footnote[1]{http://www.nestorgames.com}.

\begin{figure}[h!]
\begin{center}
\includegraphics[scale=0.5]{piece.jpg}
\caption{Exemplo da peça $3 \cdot 6$.}
\label{piece}
\end{center}
\end{figure}

Existem dois tipos de colocação de peças no Dominup:
\begin{itemize}
	\item subir - a peça é colocada em cima de duas peças adjacentes que estejam ao mesmo nível, de forma a que os números da peça colocada sejam iguais aos que ficam por baixo (um em cada peça de suporte), tal como mostra a figura~\ref{climb}.
	\item expandir - a peça é colocada na superfície de jogo, de forma a que fique adjacente e ortogonal a pelo menos uma peça já colocada, como, por exemplo, as duas peças já colocadas na figura~\ref{climb}.
\end{itemize}

\begin{figure}[h!]
\begin{center}
\includegraphics[scale=0.6]{climb.jpg}
\caption{Exemplo de um posicionamento a subir válido.}
\label{climb}
\end{center}
\end{figure}

Tal como no Dominó, as regras são relativamente simples. Começa-se por distribuir as peças aleatoriamente e de forma equilibrada pelos jogadores, mantendo a face voltada para baixo. O jogador com o duplo 7 inicia o jogo, colocando essa peça no centro da superfície de jogo e determinando a ordem dos restantes jogadores, que é dada pelo sentido contrário ao ponteiro dos relógios. Começando no segundo, cada jogador, na sua vez, realiza ambos os passos seguintes:
\begin{enumerate}
	\item Enquanto que for possível coloca peças a subir, podendo escolher a ordem em que o faz;
	\item Se ainda tiver alguma peça, coloca-a a expandir.
\end{enumerate}
Se, no final da sua vez, o jogador ficar sem peças, é declarado vencedor e o jogo termina. Alternativamente, os restantes jogadores podem continuar, de forma a determinar o segundo, terceiro e quarto lugares.

Na figura~\ref{example} pode ser observado um possível jogo de Dominup a decorrer.

\begin{figure}[h!]
\begin{center}
\includegraphics[scale=0.4]{example.jpg}
\caption{Exemplo de um jogo de Dominup.}
\label{example}
\end{center}
\end{figure}


%Descrever detalhadamente o jogo, a sua história e, principalmente, as suas regras.
%Devem ser incluidas imagens apropriadas para explicar o funcionamento do jogo.
%Devem ser incluidas as fontes de informação (e.g. URLs em rodapé).


%%%%%%%%%%%%%%%%%%%%%%%%%%
\section{Representação do Estado do Jogo}

%Descrever a forma de representação do estado do tabuleiro (tipicamente uma lista de listas), com exemplificação em Prolog de posições iniciais do jogo, posições intermédias e finais, acompanhadas de imagens ilustrativas.


%%%%%%%%%%%%%%%%%%%%%%%%%%
\section{Visualização do Tabuleiro}

Descrever a forma de visualização do tabuleiro em modo de texto e o(s) predicado(s) Prolog construídos para o efeito.
Deve ser incluída pelo menos uma imagem correspondente ao output produzido pelo predicado de visualização.


%%%%%%%%%%%%%%%%%%%%%%%%%%
\section{Movimentos}

Elencar os movimentos (tipos de jogadas) possíveis e definir os cabeçalhos dos predicados que serão utilizados (ainda não precisam de estar implementados).


\end{document}
